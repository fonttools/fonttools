\documentclass[12pt]{article}

\usepackage[letterpaper, hmargin=0.75in, vmargin=0.75in]{geometry}
\usepackage{float}
\usepackage{listings}
\usepackage{verbatim}
\usepackage[toc,page]{appendix}

\setlength{\parskip}{8pt}
\pagestyle{empty}

\title{TrueType Font Program Table Tree Shaking}

\author{
  Eduardo Hauck dos Santos\\
  \texttt{eduardohauck@gmail.com}
  \and
  Patrick Lam\\
  \texttt{p.lam@ece.uwaterloo.ca}
}
\date{}

\lstset{frame=single}

\begin{document}

\maketitle

\section{Motivation}
TrueType is an outline font standard that makes use of hinting to adjust
the display of an outline font. The executable instructions used for
hinting are called bytecode.

With the increasing popularity and spread of TrueType fonts, keeping the
contents of these fonts minimal and manageable has turned into a
challenge due to the lack of optimizing techniques for its bytecode.

More than often, fonts files in the wild contains bytecode that in
reality are never executed. This work intends to investigate how much
reduction we can obtain from a TrueType font file if we remove some of
this bytecode, specifically, unnecessary function definitions from the
font program table.

\section{Description}

The tree shaking tool is used to reduce the font program table of a
TrueType font by removing function definitions that are never called
during execution. It can be combined with subsetting techniques to
achieve a better relative reduction from the original size. After
execution, a new font file will be generated as output, with presumably
smaller size if any uncalled functions were identified during tree shaking.

To decide which functions can be eliminated, we need to identify the
functions defined and all the function calls made during execution.

Function definitions are located within the font program table in the
form of stack based code. Since it contains exclusively function
definitions, its inspection is relatively easy, as we don't have to
worry about the state of the stack being changed between each
definition. All functions are stored with their labels.

To obtain the function calls, we need to check the bytecode from every
other table, including the bytecode of each glyph. Unlike function
definitions, however, we can not obtain which labels are refered by each
call instruction straightforwardly. That is because all the values
(input for instructions) used by the bytecode are usually pushed at the
beginning of the execution. Until execution reaches to a call
instruction, previous instructions or even previous function calls
themselves can arbirtrarily change the current state of the stack.

Therefore, to identify function calls from the bytecode of other tables,
we need to keep track of how many values are pushed and popped by each
instruction and by each function called, so whenever a call instruction
is made, we know that the value at the top of the stack is the function
label we are looking for. To achieve that, we make use of the Abstract
Interpreter. During abstract execution, we can get information about the
stack height and contents after each instruction, as well as every
possible branch that can be executed, allowing us to make a list of function
calls for each table. 


\section{Background Information}

A TrueType font file is made of tables that contains data comprising an
outline font. A compreehensive list of all the tables that can be
present in a font file can be found at the Apple's TrueType Reference
Manual\cite{ttmanual}. The tables that we are going to focus on this
work are the fpgm, prep and glyf tables. The fpgm is the table that we
are trying to reduce, which contains function definitions. The prep
table contains a set of instructions that are executed whenever the font
is first accessed or when the font, point size or transformation matrix
change. The glyf table contains information that describe glyphs in the
TrueType outline format.

To gather a general idea of how much size each table takes in the font
file, we used the set of TrueType fonts Google Noto Sans\cite{notosans}
to inspect the individual size of each table for a font file. For all
fonts inspected, tables related to glyphs (e.g. glyf, GPOS, GSUB, etc)
take together more than 50 percent of the total size of whole fonts,
while the fpgm table never gets above 10 percent.  

To be able to perform the tree shaking on the font program table,
we make use of related work focused on TrueType fonts, specifically,
the TTFont class present in the fonttools project \cite{fonttools} 
(an open source project) and a bytecode manipulation library
\cite{bytecode} initiated by Wenzhu. 

The TTFont class gives us easy access to the TrueType font format. It
allows us to extract data from the font tables, perform analysis on
them, and even manipulating and saving them back to the font. 

To get information about rendered glyphs and make sure that
our modifications doesn't cause any unexpected changes in the font file,
we make use of the FreeType interpreter to load each glyph from original
and reduced fonts at a given resolution and make sure they are exactly
the same. 

The Symbolic Executor engine can be used to abstractly interpret the
program. During execution, instructions do not return actual values,
only abstract data. Although this kind of execution may offer some
limitations compared normal (concrete) interpretation, we can get an
insight on the program's control flow, including which kind of
instructions and functions were called. The Symbolic Executor does not
run using a TTFont object itself, but instead the bytecodeContainer
class, an intermediate class.

The bytecodeContainer class, as mentioned, is an intermediate of
the TTFont class. It extracts some of the contents of the a font to make
it more easily manipulable and offers the possibility of translating the
new table values back to a TTFont object. The Symbolic Executor uses a
bytecodeContainer, containing information about TTFont object to execute
tables of a font.

\section{Disregarding unneded functions}

The program starts by loading the font file as a TTFont object.
We then create a bytecodeContainer object that encapsulates the main
tables of a TrueType font file: prep, fpgm, cvt and glyf. This object 
is necessary to run the abstract executor. We also create an empty set, 
in which we will insert the labels of functions that were called at some 
point during the execution. This set will be used later on to decide 
which functions we want to keep on the font file.

Whenever a font is loaded, prep is always the first table to run to 
set the environment for glyphs to be loaded. Note that, whenever we run
a table through the abstract executor, it stores the resulting context
and information about its execution. We are interested here specifically
in function calls made during the execution.

The first step then is running the prep table through the abstract 
executor. We make a copy of the post-prep environment (we want to be
able to restore this initial state) and update our function label
set with the functions called by prep. We now have everything ready to
execute the remaining tables.

The second step is running every glyph through the abstract executor.
This is made on a loop. For each glyph we restore the execution
context post-prep, execute the glyph and update function function call
set with calls made during execution. Afterwards, we have all functions
that can possibly be called by that font.

The third and last step is obtaining the labels of the functions
to be removed (by subtracting the functions called from the list of
functions available on the fpgm table of the font). With them, we can
call the removeFunctions from the bytecodeContainer object - which will
update its function table, removing the ones passed - and finally call
updateTTFont, passing the original TTFont object. It will translate the
contents of bytecodeContainer back to the TTFont object. With that done,
we only need to call the function save from TTFont class to save the
reduced, tree shaken font.

\section{Testing}

To check if the tree shaking technique did not interfere with the
renderization of any glyphs, we developed a script to compare glyphs
from two fonts and check if they are identical. While we are using this
script to check if nothing wrong happened during the tree shaking, its
use can be extended to any sort of modification applied to the glyphs of
a TrueType Font.

To perform the comparison, we used the FreeType interpreter to render
glyphs from both fonts (original and reduced) for each charmap present
in the font. Detailed information about how the FreeType interpreter render
glyphs can be found at the FreeType Reference
Manual \cite{freetypemanual}. 

The script will compare the two fonts for a given set of glyphs at
five popular dpi resolutions: 72x72, 300x300, 600x600, 1200x1200 and
2400x2400dpi. Depending on the number of glyphs to be fetched, and from
where they have to be fetched from (file or URL), the test may take a
while. The default encoding (UTF-8) can be changed via parameter, as
well as the resolution (if a specific one must be tested)

\begin{lstlisting}
usage: pyftcompare.py [options] fontA fontB inputFileOrURL

pyftcompare -- TrueType Glyph Compare Tool

    General options:
    -h Help: print this message
    -v Verbose: be more verbose
    -e CODE Encoding: encoding used to read the input file
    -x HRES Hres: Horizontal resolution dpi
    -y VRES Vres: Vertical resolution dpi
\end{lstlisting}

\section{Results}

To measure the results of our work, we performed the tree shaking
technique on the Google Noto Sans fonts\cite{notosans}. To obtain a
better insight on how much the font program table takes in the total
size of the font, we also applied the tree shaking on  subsetted fonts
(fetching glyphs from the initial Wikipedia page from the corresponding
languages), to check against the reduction we obtained from the original
fonts.

\verbatiminput{averageStats}

As we can see from the statistics, there was a slight variation between
the reduction we obtained with the tree shaking technique for whole
fonts and for subsetted fonts. On average, we are getting double the
relative reduction when working with subsets. 

Although we obtained a better reduction for subsetted fonts, the
reduction we obtained was not very high (averaging under 10 percent). For
some fonts, like NotoSansSinhala-Regular.ttf, the difference in relative
reduction between whole and subsetted font was less than 1 percent, and
for fonts like NotoSansGeorgian-Regular.ttf, this difference jumped to
more than 10 percent. For detailed information about every font, check
Appendix A.

These results give us some insight on how much the function program
table impacts on the total size of the font, and more importantly, how
other tables contribute to the final size. We could observe that the
reduction we obtained from just subsetting the fonts greatly
varied, presenting reductions between 8 and 99 percent. The tree shaking
yields a better relative reduction in fonts that had their sizes
considerably reduced after subsetting, since the tables related to
glyphs usually take the major part of whole fonts.

For detailed information about how much each table takes in the Google
Noto Sans font files, check Appendix B.

\clearpage
\begin{appendices}
\section{Reduction Statistics}

\verbatiminput{generalStats}

\section{Tables Statatistics}


\end{appendices}

\clearpage
\begin{thebibliography}{1}

\bibitem{fonttools} The fonttools project. {\em https://github.com/behdad/fonttools } 

\bibitem{bytecode} TrueType bytecode manipulation library. 
{\em https://github.com/wenzhuman/fonttools } 

\bibitem{ttmanual} TrueType Reference Manual. 
{\em https://developer.apple.com/fonts/TrueType-Reference-Manual/ } 

\bibitem{freetypemanual} FreeType Documentation.
{\em http://www.freetype.org/freetype2/docs/documentation.html }

\bibitem{notosans} Google Noto Sans. 
{\em https://www.google.com/fonts/specimen/Noto+Sans } 

\end{thebibliography}

\end{document}
